\usepackage[hidelinks]{hyperref}   % Permite poner hipervínculos
\usepackage{url}    % Formatea texto de url correctamente
\usepackage{comment}    % Comentar bloques de líneas mediante begin-end
\usepackage{graphicx}
\usepackage[utf8]{inputenc}
\usepackage{csquotes}       % Paquete para comillas tipográficas
\usepackage{float}
\usepackage{tcolorbox}
\usepackage{amsmath}
\usepackage{geometry}
\usepackage{adjustbox}  % Para utilizar adjustbox en tablas y evitar overflow por datos
\usepackage{multicol}
\usepackage{caption}
\usepackage{lastpage}
\usepackage{fancyhdr}
\usepackage[x11names,svgnames,dvipsnames]{xcolor}
\geometry{
    a4paper,
    top = 2cm,
    left = 2cm,
    right = 2cm,
    bottom = 2cm
}

% -------- Header & Footer --------
\pagestyle{fancy}
\fancyhead[LO]{\textbf{$>$:) ICPC Reference}}
\fancyhead[C]{\leftmark\ -\ \rightmark}
\fancyhead[RO]{Page \thepage\ of \pageref{LastPage}}
\renewcommand{\headrulewidth}{0.4pt}
\fancyfoot{}
% -------------------------------------------

% -------- Frames (Cuadros flotantes) --------
\usepackage[framemethod=TikZ]{mdframed}
\usepackage{changepage}
%\usepackage[default]{sourcesanspro} % Fuente sans-serif
\usepackage{lmodern}
\newmdenv[
    backgroundcolor=gray!10,
    linecolor=black,
    roundcorner=10pt,
    linewidth=1pt,
    topline=true,
    bottomline=true,
    leftline=true,
    rightline=true,
    innertopmargin=10pt,
    innerbottommargin=10pt,
    innerleftmargin=15pt,
    innerrightmargin=15pt,
    skipabove=10pt,
    skipbelow=10pt,
]{IOFrame}

\mdfdefinestyle{terminal}{
    backgroundcolor=bashBackgroundColor,
    linecolor=white,
    roundcorner=5pt,
    skipabove=10pt,
    skipbelow=10pt,
    innerleftmargin=10pt,
    innerrightmargin=10pt,
    innertopmargin=5pt,
    innerbottommargin=5pt
}
% -------------------------------------------

% --------------- Definiendo colores ---------------
\definecolor{titleColor}{RGB}{0, 51, 153}
\definecolor{subtitleColor}{RGB}{0, 102, 204}
\definecolor{authorColor}{RGB}{204, 51, 0}
\definecolor{schoolColor}{RGB}{255, 0, 0}
\definecolor{bashLetterColor}{RGB}{180, 192, 201}
\definecolor{bashBackgroundColor}{RGB}{15, 15, 15}
% --------------------------------------------------
